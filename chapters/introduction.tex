%% introduction.tex
%%

%% ==============================
\chapter{Introduction}
\label{ch:introduction}
%% ==============================

%A reader of the introduction should be able to answer the following questions, although not in any depth.
%
%    What is the thesis about?
%    Why is it relevant or important?
%    What are the issues or problems?
%    What is the proposed solution or approach?
%    What can one expect in the rest of the thesis?
%
%State what the thesis is about early. Don't keep the reader guessing until the end of the introduction, or worse, the end of the thesis (don't laugh, I have read draft theses that left me wondering after reading the entire document). You should provide a brief and gentle overview of the thesis topic (or problem) to give the reader enough context  to understand the rest of the introduction. Don't overwhelm the reader with detail at the start. You will provide the details later elsewhere in the thesis. Target the level of writing at one of your peers, but not necessarily somebody working in the same area.
%
%State why the topic is important. Address the "so what?" criteria. Why are you working on the topic? Why should somebody else be interested? Your motivation should be obvious after the introduction, but not necessarily provably so at this point.
%
%State what the major issues are in solving your problem. Coherently overview the issues in enough detail to be able to understand they exist, but don't go into details yet or attempt to prove they exist. The overview should be in just enough depth to understand why you might propose the your particular solution or approach you are taking.
%
%Describe your proposed solution or position you're taking. Again, you should not go into minute details, nor should you attempt to prove your solution at this point; the remainder of the thesis will describe and substantiate your solution in detail, that's what a thesis is :-)
%
%At this point the reader will know what you're working on, why, what are the major issues, and what your proposed solution is, but usually only if he takes your word for it. You should outline what the reader should expect in the rest of the thesis. This is not just the table of contents in sentence form, it is an overview of the remainder of the thesis so the reader knows what to expect. 
%
%
%--------------------------------
%

\section{Motivation}



