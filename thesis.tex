\documentclass{thesisclass}
% Based on thesisclass.cls of Timo Rohrberg, 2009
% ----------------------------------------------------------------
% Thesis - Main document
% ----------------------------------------------------------------

\usepackage{caption}
\usepackage{blkarray}
\usepackage{framed}
\usepackage[linesnumbered,ruled]{algorithm2e}
\usepackage{multicol}
\usepackage{minted}
\usepackage{multicol}
\usepackage{multirow}
\usepackage{multirow}
\usepackage{booktabs}
\usepackage{environ}
\usepackage{tabularx}

%% ---------------------------------
%% | Information about the thesis  |
%% ---------------------------------

\newcommand{\myname}{Rudolf Biczok}
\newcommand{\mytitle}{Integration of internal and external gene expression and drug-perturbation data to empower novel immune therapies against Parkinson’s Disease}
\newcommand{\myinstitute}{Institute of Theoretical Computer Science}

\newcommand{\reviewerone}{Prof. Dr. Alexandros Stamatakis}
\newcommand{\reviewertwo}{Prof. Dr. Ralf Reussner}
\newcommand{\advisor}{Dr. Alexey Kozlov}

\newcommand{\timestart}{1st August 2018}
\newcommand{\timeend}{31st January 2019}

%% -------------------------------
%% |  Information for PDF file   |
%% -------------------------------

\hypersetup{
	pdfauthor={\myname},
	pdftitle={\mytitle},
	pdfsubject={Bioinformatics},
	pdfkeywords={Drug Discovery, Bioinformatics, Data Mining}
}

%% ---------------------------------
%% | Commands                      |
%% ---------------------------------

\newtheorem{definition}{Definition} \numberwithin{definition}{chapter}
\newtheorem{theorem}[definition]{Theorem}
\newtheorem{lemma}[definition]{Lemma}
\newtheorem{corollary}[definition]{Corollary}
\newtheorem{conjecture}[definition]{Conjecture}

\newcolumntype{C}{>{\centering\arraybackslash}X}

%% TODO : Is this ever used?
\NewEnviron{myTable}[4]{%
	\vspace{5px}%
	\centering%
	\rowcolors{2}{black!25}{white}%
	\captionsetup{type=table}
	\begin{tabular}{|#2|}%
		\arrayrulecolor{black}%
		\hline%
		\BODY
		\hline%
	\end{tabular}%
	\vspace{-6px}%
	\captionof{table}[#3]{#3#1} \label{#4}%
	\vspace{5px}
}

\NewEnviron{centeredFigure}[1][]{%
	\begin{figure}[#1]
		\centering
		\BODY
	\end{figure}	
}

\newcommand*\circled[1]{\tikz[baseline=(char.base)]{
		\node[shape=circle,draw,inner sep=2pt, fill=black] (char) {\textcolor{white}{#1}};}}

\newcommand{\thc}[1]{\textbf{\textcolor{white}{#1}}}

\newcommand{\myHeaderCell}[1]{\cellcolor{black}\thc{#1}}

\newcommand{\myMultiLineCell}[2][c]{%
	\begin{tabular}{@{}#1@{}}#2\end{tabular}%
}

\newcommand{\cbrac}[1]{\lbrace #1\rbrace}

%% --------------------------------
%% | Settings for word separation |
%% --------------------------------
% Help for separation:
% In german package the following hints are additionally available:
% "- = Additional separation
% "| = Suppress ligation and possible separation (e.g. Schaf"|fell)
% "~ = Hyphenation without separation (e.g. bergauf und "~ab)
% "= = Hyphenation with separation before and after
% "" = Separation without a hyphenation (e.g. und/""oder)

% Describe separation hints here:
\hyphenation{
% Pro-to-koll-in-stan-zen
% Ma-na-ge-ment  Netz-werk-ele-men-ten
% Netz-werk Netz-werk-re-ser-vie-rung
% Netz-werk-adap-ter Fein-ju-stier-ung
% Da-ten-strom-spe-zi-fi-ka-tion Pa-ket-rumpf
% Kon-troll-in-stanz
}

%Break links in URLs
\def\UrlBreaks{\do\/\do-}


%% ------------------------
%% |    Including files   |
%% ------------------------
% Only files listed here will be included!
% Userful command for partially translating the document (for bug-fixing e.g.)
\includeonly{
titlepage,
chapters/introduction,
chapters/sota,
chapters/conclusion,
chapters/appendix
}


%%%%%%%%%%%%%%%%%%%%%%%%%%%%%%%%%
%% Here, main documents begins %%
%%%%%%%%%%%%%%%%%%%%%%%%%%%%%%%%%
\begin{document}

% Add common save boxes
\newsavebox{\saveBoxOne}
\newsavebox{\saveBoxTwo}
\newsavebox{\saveBoxThree}
\newsavebox{\saveBoxFour}
\newsavebox{\saveBoxFive}
\newsavebox{\saveBoxSix}
\newsavebox{\saveBoxSeven}
\newsavebox{\saveBoxEight}

% Remove the following line for German text
\selectlanguage{english}

\frontmatter
\pagenumbering{roman}
\include{titlepage}
\blankpage

%% -------------------------------
%% |   Statement of Authorship   |
%% -------------------------------

\thispagestyle{plain}

\vspace*{\fill}

\centerline{\textbf Statement of Authorship}

\vspace{0.25cm}

I hereby declare that this document has been composed by myself and describes my own work, unless otherwise acknowledged in the text.

\vspace{2.5cm}

\hspace{0.25cm} Heidelberg, 31st January 2019

\vspace{2cm}

\blankpage

%% -------------------
%% |   Abstract      |
%% -------------------

\thispagestyle{plain}

\begin{addmargin}{0.5cm}

\centerline{\textbf Abstract}

%A short summary of what is going on here.

Bla bla

\vskip 2cm

\centerline{\textbf Deutsche Zusammenfassung}

%Kurze Inhaltsangabe auf deutsch.

Bla bla

\vskip 2cm

\newpage

\centerline{\textbf Acknowledgements}

Bla Bla

\end{addmargin}

\blankpage

%% -------------------
%% |   Directories   |
%% -------------------

\tableofcontents
\blankpage

%% -----------------
%% |   Main part   |
%% -----------------

\mainmatter
\pagenumbering{arabic}
%% introduction.tex
%%

%% ==============================
\chapter{Introduction}
\label{ch:introduction}
%% ==============================

%A reader of the introduction should be able to answer the following questions, although not in any depth.
%
%    What is the thesis about?
%    Why is it relevant or important?
%    What are the issues or problems?
%    What is the proposed solution or approach?
%    What can one expect in the rest of the thesis?
%
%State what the thesis is about early. Don't keep the reader guessing until the end of the introduction, or worse, the end of the thesis (don't laugh, I have read draft theses that left me wondering after reading the entire document). You should provide a brief and gentle overview of the thesis topic (or problem) to give the reader enough context  to understand the rest of the introduction. Don't overwhelm the reader with detail at the start. You will provide the details later elsewhere in the thesis. Target the level of writing at one of your peers, but not necessarily somebody working in the same area.
%
%State why the topic is important. Address the "so what?" criteria. Why are you working on the topic? Why should somebody else be interested? Your motivation should be obvious after the introduction, but not necessarily provably so at this point.
%
%State what the major issues are in solving your problem. Coherently overview the issues in enough detail to be able to understand they exist, but don't go into details yet or attempt to prove they exist. The overview should be in just enough depth to understand why you might propose the your particular solution or approach you are taking.
%
%Describe your proposed solution or position you're taking. Again, you should not go into minute details, nor should you attempt to prove your solution at this point; the remainder of the thesis will describe and substantiate your solution in detail, that's what a thesis is :-)
%
%At this point the reader will know what you're working on, why, what are the major issues, and what your proposed solution is, but usually only if he takes your word for it. You should outline what the reader should expect in the rest of the thesis. This is not just the table of contents in sentence form, it is an overview of the remainder of the thesis so the reader knows what to expect. 
%
%
%--------------------------------
%

\section{Motivation}




%% background.tex
%%

%% ==============================
\chapter{State of the art}
\label{ch:sota}
%% ==============================

\section{Gene Expression Analysis}

The most recurring task in pharmaceutical research \& early development is the gene expression analysis on a given date source. 

\subsection{Methods for differential gene expression inferrence}



Name | Strategy | Prefered Input Data
-----|------------|--------------------
[edgeR](http://doi.org/10.1093/bioinformatics/btp616) | Negative binomial distribution + Trimmed Mean of M values (TMM) Normalization | RNAseq
[DEseq](https://doi.org/10.1186/s13059-014-0550-8) | Negative binomial distribution + scaling factor normalization procedure | RNAseq
[limma](https://doi.org/10.1093/nar/gkv007) | Linear Modeling + voom transformation of counts (vor RNAseq) | RNAseq \& microarray

The "rule of thumb" is to use limma for microarray data and edgeR for RNAseq data. DEseq is used to verify that a hypothesis based on edgeR results can also be derived from DEseq results (since edgeR is known to report more false positives).

\subsection{Methods for batch-effect correction in meta-analysis}

We use ComBat, SVA, and BioQC. Here we will probably have to compare different methods and reach a conclusion.

\subsection{Methods for gene-set/pathway enrichment analysis}

We use CAMERA, BioQC, and Fisher's exact test.
Previously we found out that CAMERA and BioQC will lead to false negatives when many genes are differentially expressed, while Fisher's exact test will not (not published). We can verify this and make a meta-method to accommodate different scenarios.

\section{ROGER}

\subsection{ROGER Database}

* **Annotations**
* Gene Annotation (consumes biomart)
* GeneAnnotation: Ensemble \& NCBI gen IDs. ROGER-internal GeneIndex, Gen meta data
* Orthologs: Mapping between orthologous genes between different species
* TranscriptioAnnotation: Ensemble Transcription ID and meta data
* TranscriptRefSeq: NCBI Transcription ID and meta data
* Genesets (consumes mongodb/json \& gmt)
* DefaultGenesets: Available gen set data
* DefaultGenesetCategory: For gen set categorization
* DefaultGenesets2gene: Mapping of gen set data to gen annotations
* **Input**
* Datasets: Raw expression data
* Phenodata
* Designs: Relevant Feature matrix
* Contrasts: Contrast matrix
* **Methods \& Results**
* GSEmethods: Used Gen enrichment method (e.g. CAMERA)
* GSEtables: Gen enrichment results
* DGEmethods: Used Differential Gen Expression inference method (e.g.  edgeR, limma)
* DGEmodels: Used DGE model based on Desing and Cntrast information
* DGEtables: Results from DEG inference

\subsection{Annotation Problems}

* Have to support both Ensembl and NCBI IDs
* Ensembl has many unconsistent / deprecated data: Some Gene Symbols apper in multiple EnsembleGeneIds, 
* Possible fix: pick the "most accurate on" (e.g. does it have a proper chromosone? Number of Transcripts etc.)

% ![ROGER Schema](roger/old_schema.png)

\subsection{RESTful APIs for scientific R pipelines}

* [rplumber](https://www.rplumber.io/)
* Fastest way to deploy REST services
* Very low-level: No load balancing, no authentication, task management, ...
* [OpenCPU](https://www.opencpu.org)
* Load balancing
* Lightwing WEB API basedn on JavaScript
* No build-in support for [long running jobs]
(https://github.com/opencpu/opencpu/issues/141)
* No build-in task management
* [Flask](http://flask.pocoo.org)
* Python equivalent to OpenCPU
* Established in the department
* Requires wapper functions between python <-> R
* No build-in task management

\section{Differential Gene Expression}

Lets assume we have a study consisting of a set of samples $D = \cbrac{A, B, C, D, E, F}$. Samples $A$ and $B$ are from macrophage cells, $C$ and $D$ are from microglia cells, and $E$ and $F$ are from monocyte cells. 

Then we have basically two ground ways to model the experiments:

\begin{lrbox}{\saveBoxOne}
	\mintinline{R}{model.matrix(~CellType)} 
\end{lrbox}

\begin{lrbox}{\saveBoxTwo}
	\mintinline{R}{model.matrix(~0+CellType)}
\end{lrbox}

\begin{lrbox}{\saveBoxThree}
\begin{minipage}{5cm}
	\[
		\begin{blockarray}{cccc}
			& \mu & \text{\rotatebox{90}{micro}} & \text{\rotatebox{90}{mono}} &  \\
			\begin{block}{c(ccc)}
				A & 1 & 0 & 0  \\
				B & 1 & 0 & 0  \\
				C & 1 & 1 & 0  \\
				D & 1 & 1 & 0  \\
				E & 1 & 0 & 1  \\
				F & 1 & 0 & 1  \\
			\end{block}
		\end{blockarray}
	\]
\end{minipage}
\end{lrbox}

\begin{lrbox}{\saveBoxFour}
\begin{minipage}{5cm}
	\[
		\begin{blockarray}{cccc}
			& \text{\rotatebox{90}{macro}} & \text{\rotatebox{90}{micro}} & \text{\rotatebox{90}{mono}} &  \\
			\begin{block}{c(ccc)}
				A & 1 & 0 & 0  \\
				B & 1 & 0 & 0  \\
				C & 0 & 1 & 0  \\
				D & 0 & 1 & 0  \\
				E & 0 & 0 & 1  \\
				F & 0 & 0 & 1  \\
			\end{block}
		\end{blockarray}
	\]
\end{minipage}
\end{lrbox}

\begin{lrbox}{\saveBoxFive}
	\begin{minipage}{5.3cm}
		\[ y \thicksim \mu + \beta_{\text{micro}} x_{\text{micro}} + \beta_{\text{mono}} x_{\text{mono}}\] 
	\end{minipage}
\end{lrbox}

\begin{lrbox}{\saveBoxSix}
	\begin{minipage}{5.3cm}
		\[ \begin{split}
		y \thicksim \ & \beta_{\text{macro}} x_{\text{macro}} + \beta_{\text{micro}} x_{\text{micro}} \\
		& \beta_{\text{mono}} x_{\text{mono}}
		\end{split} \]
	\end{minipage}
\end{lrbox}

\begin{lrbox}{\saveBoxSeven}
			\begin{minipage}{5.3cm}
				\[
					\begin{blockarray}{ccc}
						\begin{block}{c(cc)}
						\mu          & 0 & 0 \\
						\text{micro} & 1 & 0 \\
						\text{mono}  & 0 & 1 \\
						\end{block}
					\end{blockarray}
				\]
		\end{minipage}
\end{lrbox}

\begin{lrbox}{\saveBoxEight}
	\begin{minipage}{5.3cm}
			\[
			\begin{blockarray}{cccc}
			\begin{block}{c(ccc)}
			\text{macro} & -1 & -1 & 0\\
			\text{micro} & 1 & 0 & -1 \\
			\text{mono}  & 0 & 1 & 1 \\
			\end{block}
			\end{blockarray}
			\]
	\end{minipage}
\end{lrbox}

\begin{table}[ht]

	\begin{tabularx}{\linewidth}{|lCC|}%
		\arrayrulecolor{black}%
		\hline
	
		\rowcolor{black} & \thc{1 vs Average} & \thc{1 vs 1} \\
	
		\circled{1} & \usebox{\saveBoxOne} & \usebox{\saveBoxTwo} \\
		\hline%
	
		\circled{2} & \usebox{\saveBoxThree} & \usebox{\saveBoxFour} \\
		
		\circled{3} & \usebox{\saveBoxFive} 
		& \usebox{\saveBoxSix}  \\

		\circled{4} & \usebox{\saveBoxSeven} 
		& \usebox{\saveBoxEight}  \\
		\hline%
		
		\hline
	\end{tabularx}

	\caption{Overview of example experiments} \label{fig:example_experiments}
\end{table}

%\include{TODO}
%% conclusion.tex
%%

%% ==================
\chapter{Conclusion and Future Work}
\label{ch:conclusion}
%% ==================

%Recap on your thesis. It has been a long journey if the reader has made it this far. Remind the reader what the big picture was. Briefly outline your thesis, motivation, problem, and proposed solution.

%Now the most important part, draw conclusions based on your analysis. Did your proposed solution work? What are the strong points? What are the limitations?

%Significant issues identified in the thesis, or still outstanding after the thesis, should be describe as future work.



%% --------------------
%% |   Bibliography   |
%% --------------------

\cleardoublepage
\phantomsection
\addcontentsline{toc}{chapter}{\bibname}

\iflanguage{english}
{\bibliographystyle{alpha}}
{\bibliographystyle{babalpha-fl}} % german style

\bibliography{references}


%% ----------------
%% |   Appendix   |
%% ----------------

\cleardoublepage
\input{chapters/appendix}


\end{document}
